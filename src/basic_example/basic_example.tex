\documentclass[10pt,a4paper,twoside]{article}
% IMPORTS AND LIBRARIES
\usepackage[utf8]{inputenc}
\usepackage[french]{babel}
\usepackage{amsmath}
\usepackage{amsfonts}
\usepackage{amssymb}
\usepackage{graphicx}
\usepackage{hyperref}
% \usepackage{booktabs}
\usepackage[math]{blindtext} % generates text for document filling purposes
% \usepackage{lipsum}
\usepackage[left=4cm,right=4cm,top=2cm,bottom=2cm]{geometry}
\usepackage{siunitx}
\usepackage{fancyhdr}
\pagestyle{fancy}
%... then configure it.
\fancyhead{} % clear all header fields
\fancyhead[RO,LE]{\textbf{The performance of new graduates}}
\fancyfoot{} % clear all footer fields
\fancyfoot[LE,RO]{\thepage}
\fancyfoot[LO,CE]{From: K. Grant}
\fancyfoot[CO,RE]{To: Dean A. Smith}
% DOCUMENT SETUP
\author{Ludovic Charleux}
\title{My first ever LaTeX document}
\date{May 5th, 2025}
% DOCUMENT
\begin{document}
\maketitle

\tableofcontents % Affichage de la table des matières

\section{Introduction}

Ceci est le corps de mon document.
Je peux écrire du texte ici, et il sera affiché dans le document final.

\blindtext[1]


%\pagebreak

\section{Maths}

\subsection{Les maths de base}

On peut écrire des équations:
\blindtext

$$
    a x + b = 5
$$

\noindent On peut utiliser les symboles mathématiques:

\begin{equation}
    u_n = \int_{x=0}^\infty v^2(x) dx
    \label{eq:nice_one}
\end{equation}

On voit dans l'équation Eq. \ref{eq:nice_one} \footnote{voir page \pageref{eq:nice_one}} que la théorie est fausse.

\subsection{Les maths en ligne}

On peut intégrer des maths dans le texte, par exemple $\alpha = 5$ et $x = \frac{3}{4}$.
On remarque que ces résultats sont très importants.
On remarque que ces résultats sont très importants.

\subsection{Des maths un peu plus avancées}

On veut montrer des choses:

\begin{align}
    A & = \left( a + b\right)^2 \nonumber \\
      & = a^2 + 2 ab + b^2
\end{align}

On vous donne les valeurs suivantes:
\begin{itemize}
\item La masse $m = \qty{5.01d23}{\kilo\gram}$.
\item La vitesse 
$$
v = \qty[per-mode=fraction]{5}{\meter\per\second}$$
\item La pression:
$$
P = \qty{5}{\newton\per\meter\squared}
$$
\end{itemize}

\section{Enumerations}

On peut énumérer:

\begin{enumerate}
\item Point 1
\item Point 2
\end{enumerate}

\section{Mise en page}

On peut utiliser FancyHDR pour modifier la mise en page du documen. 
Pour aller plus loin, voir \href{https://fr.overleaf.com/learn/latex/Headers_and_footers#Using_the_fancyhdr_package}{Overleaf}.



\end{document}